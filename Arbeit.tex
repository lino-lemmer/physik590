\documentclass[
    11pt,
    ngerman
]{scrreprt}


% Zeilenumbrüche

\parindent 0pt
\parskip 6pt

% Für deutsche Buchstaben und Synthax

\usepackage[ngerman]{babel}

% Für Auflistung mit speziellen Aufzählungszeichen

\usepackage{paralist}

% zB für \del, \dif und andere Mathebefehle

\usepackage{amsmath}
\usepackage{commath}
\usepackage{amssymb}

% Für Literatur/bibliography

\IfFileExists{bibliography.bib}{
    \newcommand{\bibliographyfile}{bibliography.bib}
}{
    \newcommand{\bibliographyfile}{../../bibliography.bib}
}


\usepackage[
    backend=biber,
    style=alphabetic,
    hyperref=true
]{biblatex}

\IfFileExists{\bibliographyfile}{
    \bibliography{\bibliographyfile}
    \AtEndDocument{\printbibliography}
}{}
    
% Für \SIunit[]{} und \num in deutschem Stil

 \usepackage[output-decimal-marker={,}]{siunitx}
 \DeclareSIUnit\clight{\ensuremath{c}}

% Schriftart und encoding

\usepackage[utf8]{inputenc}
% Bitstream charter als default
\usepackage[charter, greekuppercase=italicized]{mathdesign}
% Lato, als sans default
\renewcommand{\sfdefault}{fla}

% Für \sfrac{}{}, also inline-frac

\usepackage{xfrac}

% Für Einbinden von pdf-Grafiken

\usepackage{graphicx}

% Zellen in Tabelle verbinden

\usepackage{multirow}

% einzelne Querformat-Seiten

\usepackage{pdflscape}

% TikZ

\usepackage{tikz}
\usetikzlibrary{arrows}
\usetikzlibrary{calc}
\usetikzlibrary{decorations.pathmorphing}
\usepackage{pgfplots}

\tikzset{
    wave/.style={decorate, decoration=snake}
}

% Umfließen von Bildern

% \usepackage{floatflt}

% Für weitere Farben

\usepackage{color}

% Für Streichen von z.B. $\rightarrow$

\usepackage{centernot}

% Für Befehl \cancel{}

\usepackage{cancel}
\newcommand\ccancel[2][black]{\renewcommand\CancelColor{\color{#1}}\cancel{#2}}

% Für Links nach außen und innerhalb des Dokumentes

\usepackage{hyperref}

% Für Layout von Links

\hypersetup{
	citecolor=black,
	colorlinks=true,
	linkcolor=black,
	urlcolor=blue,
}


\newcommand\ignore[1]{}

% Verschiedene Mathematik-Hilfen

% Manuelles taggen z.B. in align*-Umgebung

\newcommand\mantag{\stepcounter{equation}\tag{\theequation}}

\newcommand \e[1]{\cdot10^{#1}}
\newcommand\p{\partial}

\newcommand\half{\frac 12}
\newcommand\shalf{\sfrac12}

\newcommand\skp[2]{\left\langle#1,#2\right\rangle}
\newcommand\mw[1]{\left\langle#1\right\rangle}

\newcommand \ee{\mathrm e}
\newcommand \eexp[1]{\mathrm{e}^{#1}}
\newcommand \dexp[1]{\exp\left({#1}\right)}

% Trigonometrische Funktionen mit Argument in Klammern

\newcommand \dsin[1]{\sin\left({#1}\right)}
\newcommand \dcos[1]{\cos\left({#1}\right)}
\newcommand \dtan[1]{\tan\left({#1}\right)}
\newcommand \darccos[1]{\arccos\left({#1}\right)}
\newcommand \darcsin[1]{\arcsin\left({#1}\right)}
\newcommand \darctan[1]{\arctan\left({#1}\right)}

\newcommand{\ui}[1]{\int_{-\infty}^{\infty}\dif {#1}\;}

% Für fette, serifenlose Matrix

\newcommand \mat[1]{\mathbf{#1}}

% Nabla und Kombinationen von Nabla

\renewcommand\div[1]{\skp{\nabla}{#1}}
\newcommand\rot{\nabla\times}
\newcommand\grad[1]{\nabla#1}
\newcommand\laplace{\triangle}
\newcommand\dalambert{\mathop{{}\Box}\nolimits}

%Für komplexe Zahlen

\newcommand \ii{\mathrm i}
\renewcommand{\Im}{\mathop{{}\mathrm{Im}}\nolimits}
\renewcommand{\Re}{\mathop{{}\mathrm{Re}}\nolimits}

%Für Bra-Ket-Notation

\newcommand\bra[1]{\left\langle#1\right|}
\newcommand\ket[1]{\left|#1\right\rangle}
\newcommand\braket[2]{\left\langle#1\left.\vphantom{#1 #2}\right|#2\right\rangle}
\newcommand\braopket[3]{\left\langle#1\left.\vphantom{#1 #2 #3}\right|#2\left.\vphantom{#1 #2 #3}\right|#3\right\rangle}


\author{Lino Lemmer}
\subject{Bachelorarbeit in Physik }
\title{Brustkrebsdiagnose durch ultraschallinduzierte Gewebeverschiebung}
\subtitle{Optimierung von Messverfahren und Bildertrag}
\publishers{Vorgelegt der Rheinischen Friedrich-Wilhelms-Universität Bonn}



\begin{document}
\maketitle


\tableofcontents

\chapter{Einleitung}

Brustkrebs ist mit über 75000 Neuerkrankungen in Deutschland, allein im Jahr 2014, die häufigste bösartige Tumorerkrankung \parencite[68]{krebs_in_deutschland}.

\chapter{Physikalische Grundlagen}

\section{Ultraschall}

\subsection{Grundlagen}

\subsection{Erzeugung}

\section{Magnet-Resonanz-Tomographie}

\subsection{Kernspinresonanz}

\subsection{Relaxation}

\subsection{Spin-Echo-Sequenzen}

\subsection{Ortsauflösung}

\section{Versuchsvorrichtung}

\begin{figure}
    \centering
    \begin{tikzpicture}
        % PC
        \draw (0,0) rectangle (1,1);
        \draw (-.2,-.2) rectangle (1.2,1.2);
        \draw (-.2,-.2) -- (-.4,-.4) -- (1.4,-.4) -- (1.2,-.2);
        \node at (.5,.5) {PC};

        % Kabel
        \draw[out=0, in=180]
            (1.2,.4) to (3,-.55)
            (1.2,.5) to (3,.45)
            (1.2,.6) to (3.,1.45);

        % Steuerbox mit Schrittmotoren
        \draw (3.,-1.1) rectangle (7,2);
        \foreach \y in {-1.0,0,1}
        \draw (3,\y) rectangle node{1} (4,\y+.9);

        % Kolben
        % Kolben 1
        \draw (4, -.7) rectangle (5.5,-.4);
        \draw[ultra thick] (4,-.55) -- (4.5,-.55);
        \draw (4.5,-.7) rectangle (4.6,-.4);

        % Kolben 2
        \draw (4, .3) rectangle (6.8,.6);
        \draw[ultra thick] (4,.45) -- (4.5,.45);
        \draw (4.5,.3) rectangle (4.6,.6);

        % Kolben 3
        \draw (4, 1.3) rectangle (5.,1.6);
        \draw[ultra thick] (4,1.45) -- (4.5,1.45);
        \draw (4.5,1.3) rectangle (4.6,1.6);

        % Verschiebevorrichtung
        \draw (8.7,-.2) rectangle (11.8,1.1);

        % Kolben 2
        \draw (8.8,.3) rectangle (11.6,.6);
        \draw[ultra thick] (9.4,.45) -- (12,.45);
        \draw (9.3,.3) rectangle (9.4,.6);

        % Kolben 1
        \draw (12,1.2) rectangle (12.3,-.3);
        \draw[ultra thick] (12.15,.6) -- (12.15,-.7);
        \draw (12,.7) rectangle (12.3,.6);

        % Kolben 3
        \draw (12.4,.9) rectangle (13.4,1.2);
        \draw[ultra thick] (13.0,1.05) -- (13.8,1.05);
        \draw (12.9,.9) rectangle (13.0,1.2);

        % Verbindung Kolben 1, 3 Emitter und Spiegel
        \draw (12.1,-.7) -- (12.4,-.7) -- (12.4,1.2);
        \draw (12.4,.5) -- (12.5,.5) coordinate (a) ;
        \draw (12.4,.8) --  (12.5,.8) coordinate (b);
        \draw[bend left] (b) to (a);
        \draw  ($(b)!.5!(a)$) -- (12.7,0) node[below]{4} ;
        \draw (13.8, 1.2) -- (13.8, 0.9);
        \draw (13.7,.9) rectangle coordinate (spiegel) (13.9,.4);
        \draw (spiegel) -- (13.5,0) node[below]{5};

        % Schläuche mit Steckverbindung und Kupplungen

        \draw (5.4,-.4) to [out=90, in=-90] (6.9,.4) to [out=90, in=-90] (6.6,2) to [out=90, in=180] (7.6,2.6) -- (8.,2.6);
        \draw[ultra thick] (7.6,2.6) -- coordinate (S1)  (7.8,2.6) ;
        \draw (8.,2.5) rectangle coordinate (K1) (8.3,2.7);
        \draw (8.3,2.3) -- (8.5,2.3);
        \draw (8.6,2.3) to [out=0, in=90] (9,.6);

        \draw (6.6,.6) to [out=90, in=-90] (6.5,2) to [out=90, in=180] (7.6,2.3) -- (8.,2.3);
        \draw[ultra thick] (7.6,2.3) --  coordinate (S2) (7.8,2.3) ;
        \draw (8.,2.4) rectangle coordinate (K2) (8.3,2.2);
        \draw (8.3,2.6) -- (8.6,2.6);
        \draw (8.7,2.6) to [out=0, in=90] (12.15,1.2);
        
        \draw (4.9,1.6) to [out=90, in=-90] (6.4,2) to [out=90, in=180] (7.6,2.9) -- (8.,2.9);
        \draw[ultra thick] (7.6,2.9) -- coordinate (S3) (7.8,2.9);
        \draw (8.,2.8) rectangle coordinate (K3) (8.3,3.0);
        \draw (8.3,2.9) -- (8.7,2.9);
        \draw (8.8,2.9) to [out=0, in=90] (12.6,1.2);

        \draw (8.45,2.2) -- (8.75,3.0);
        \draw (8.55,2.2) -- (8.85,3.0);

        \node (S) at (7.5,3.8) {2};
        \foreach \y in {(S1), (S2), (S3)}
            \draw (S) -- \y;

        \node (K) at (8.3,3.8) {3};
        \foreach \y in {(K1), (K2), (K3)}
            \draw (K) -- \y;

        % Druckausgleich

            \draw (13.,-2) rectangle node {6} (13.75,-1);

            \draw (12,-.2) to [out=180, in=180] (13,-1.3);
            \draw (11.5,.3) to [out=-90, in=180] (13,-1.4); 
            \draw (13.3,.9) to [out=-90, in=180] (13,-1.2);

        % Beschriftung
        \node at (0,4) {\parbox{4.5cm}{
                1: Schrittmotor \\
                2: Steckverbindung \\
                3: Kupplung, tropffrei \\
                4: Ultraschallemitter \\
                5: Spiegel \\
                6: Druckausgleichsbehälter
            }
        };
    \end{tikzpicture}
    \caption{%
        Verschiebevorrichtung
    }
    \label{fig:aufbau_total}
\end{figure}


\chapter{Durchführung}

\section{Verbesserung der Versuchsvorrichtung}

An der vorhandenen Versuchsvorrichtung bestanden zu Anfang meiner Arbeit einige
Probleme die es zu verbessern galt. Zum Einen ließen sich die Kolben nach
einiger Zeit des Stehens nur sehr schwergängig in ihren Zylindern bewegen.
Damit die Schrittmotoren nicht überfordert würde musste man diese vor der
Benutzung per Hand lockern. Zum Anderen trat Luft bei Benutzung der Apperatur
in das System ein, wodurch nach einigem Fahren die Eichung der Verschiebung
nicht mehr stimmte.  

\subsection{Problem: Dichtungsringe}

Zum abdichten waren auf den Kolben zwei O-Ringe in jeweils eine Nut
eingelassen. Diese Nuten waren nur minimal breiter als die Schnurstärle der
Ringe. Befand sich der Kolben nun im Zylinder wurden die Ringe stark verformt
(Abbildung \ref{fig:nut_schmal}). Die dadurch entstehende breite Auflagefläche
und starke Rückstellkraft sorgt für hohe Reibung. Ist die Nut breiter kann sich
der Ring besser verformen (Abbildung \ref{fig:nut_breit}). Wird zudem noch die
Nut tiefer gestochen, verringert sich die Verformung und damit die
Auflagefläche und Rückstellkraft deutlich. Die Reibung schrumpft.

Da wir bei unserem Versuchsaufbau mit geringen Drücken zu tun haben, reicht
zudem eine Abdichtung mit nur einem O-Ring pro Kolben, statt der vorher
verwendeten zwei.

\begin{figure}
\begin{minipage}[htbp]{.45\textwidth}
    \centering
        \begin{tikzpicture}
            \draw (.5,0) -- (2,0) -- (2,-2.5) -- (5,-2.5) -- (5,0) -- (6.5,0) ;
            \draw (3.5,-1.05) circle (1.43);
            \draw[red] (.5,.07) -- (6.5,.07);
            \foreach \x in {2.7, 3.5, 4.3} \draw[ultra thick, ->, red] (\x,-1) -- (\x,0.05);
            \draw[red] (2.7,.05) 
            -- (4.3,.05) 
            arc (90:0:.68) 
            -- (4.98,-1.8) 
            arc (0:-90:.68)
            -- (2.7,-2.48)
            arc (-90:-180:.68)
            -- (2.02,-.63)
            arc (-180:-270:.68);
        \end{tikzpicture}
        \caption{%
            Breite Auflagefläche des O-Rings bei zu schmaler Nut.
        }
        \label{fig:nut_schmal}
\end{minipage}
\hfill
\begin{minipage}[htbp]{.45\textwidth}
    \centering
        \begin{tikzpicture}
            \draw (1,0) -- (2,0) -- (2,-2.5) -- (6,-2.5) -- (6,0) -- (7,0) ;
            \draw (4,-1.05) circle (1.43);
            \draw[red] (1,0.07) -- (7,0.07);
            \foreach \x in {3.5, 4.5} \draw[ultra thick, ->, red] (\x,-1) -- (\x,0.05);
            \draw[red] (3.5,.05) 
            -- (4.5,.05) 
            arc (90:-90:1.26) 
            -- (3.5,-2.48)
            arc (-90:-270:1.26);
        \end{tikzpicture}
        \caption{%
            Kleine Auflagefläche wenn die Nut deutlich breiter ist als der O-Ring.
        }
        \label{fig:nut_breit}
\end{minipage}
\end{figure}

Ein weiterer Faktor ist die Wahl des Schmiermittels. Bisher wurde das
Silikonbasierte High Vacuum Grease verwendet. Ich hab mich für eine Verwendung
von Hochleistungs \textsc{PTFE}-Spray entschieden, da dieses auch nach längerem
Stehen unter Druck gute Laufeigenschaften hat.


\subsection{Problem: Kupplung}

\subsection{Kalibrierung}

\subsection{Messungen}

\chapter{Ergebnis/Ausblick}
\IfFileExists{\bibliographyfile}{
    \printbibliography
}{}

\end{document}

%vim: tw=79
