% Tikzpicture Verschiebevorrichtung
    \begin{tikzpicture}
        % PC
        \draw (0,0) rectangle (1,1);
        \draw (-.2,-.2) rectangle (1.2,1.2);
        \draw (-.2,-.2) -- (-.4,-.4) -- (1.4,-.4) -- (1.2,-.2);
        \node at (.5,.5) {PC};

        % Kabel
        \draw[out=0, in=180]
            (1.2,.4) to (3,-.55)
            (1.2,.5) to (3,.45)
            (1.2,.6) to (3.,1.45);

        % Steuerbox mit Schrittmotoren
        \draw (3.,-1.1) rectangle (7,2);
        \foreach \y in {-1.0,0,1}
        \draw (3,\y) rectangle node{1} (4,\y+.9);

        % Kolben
        % Kolben 1
        \draw (4, -.7) rectangle (5.5,-.4);
        \draw[ultra thick] (4,-.55) -- (4.5,-.55);
        \draw (4.5,-.7) rectangle (4.6,-.4);

        % Kolben 2
        \draw (4, .3) rectangle (6.8,.6);
        \draw[ultra thick] (4,.45) -- (4.5,.45);
        \draw (4.5,.3) rectangle (4.6,.6);

        % Kolben 3
        \draw (4, 1.3) rectangle (5.,1.6);
        \draw[ultra thick] (4,1.45) -- (4.5,1.45);
        \draw (4.5,1.3) rectangle (4.6,1.6);

        % Verschiebevorrichtung
        \draw (8.7,-.2) rectangle (11.8,1.1);

        % Kolben 2
        \draw (8.8,.3) rectangle (11.6,.6);
        \draw[ultra thick] (9.4,.45) -- (12,.45);
        \draw (9.3,.3) rectangle (9.4,.6);

        % Kolben 1
        \draw (12,1.2) rectangle (12.3,-.3);
        \draw[ultra thick] (12.15,.6) -- (12.15,-.7);
        \draw (12,.7) rectangle (12.3,.6);

        % Kolben 3
        \draw (12.4,.9) rectangle (13.4,1.2);
        \draw[ultra thick] (13.0,1.05) -- (13.8,1.05);
        \draw (12.9,.9) rectangle (13.0,1.2);

        % Verbindung Kolben 1, 3 Emitter und Spiegel
        \draw (12.1,-.7) -- (12.4,-.7) -- (12.4,1.2);
        \draw (12.4,.5) -- (12.5,.5) coordinate (a) ;
        \draw (12.4,.8) --  (12.5,.8) coordinate (b);
        \draw[bend left] (b) to (a);
        \draw  ($(b)!.5!(a)$) -- (12.7,0) node[below]{4} ;
        \draw (13.8, 1.2) -- (13.8, 0.9);
        \draw (13.7,.9) rectangle coordinate (spiegel) (13.9,.4);
        \draw (spiegel) -- (13.5,0) node[below]{5};

        % Schläuche mit Steckverbindung und Kupplungen

        \draw[double] (5.4,-.4) to [out=90, in=-90] (6.9,.4) to [out=90, in=-90] (6.6,2) to [out=90, in=180] (7.6,2.6) -- (8.,2.6);
        \draw[ultra thick] (7.6,2.6) -- coordinate (S1)  (7.8,2.6) ;
        \draw (8.,2.5) rectangle coordinate (K1) (8.3,2.7);
        \draw[double] (8.3,2.6) -- (8.55,2.6);
        \draw[double] (8.65,2.6) to [out=0, in=90] (12.15,1.2);

        \draw[double] (6.6,.6) to [out=90, in=-90] (6.5,2) to [out=90, in=180] (7.6,2.3) -- (8.,2.3);
        \draw[ultra thick] (7.6,2.3) --  coordinate (S2) (7.8,2.3) ;
        \draw (8.,2.4) rectangle coordinate (K2) (8.3,2.2);
        \draw[double] (8.3,2.3) -- (8.51,2.3);
        \draw[double] (8.61,2.3) to [out=0, in=90] (9,.6);
        
        \draw[double] (4.9,1.6) to [out=90, in=-90] (6.4,2) to [out=90, in=180] (7.6,2.9) -- (8.,2.9);
        \draw[ultra thick] (7.6,2.9) -- coordinate (S3) (7.8,2.9);
        \draw (8.,2.8) rectangle coordinate (K3) (8.3,3.0);
        \draw[double] (8.3,2.9) -- (8.59,2.9);
        \draw[double] (8.69,2.9) to [out=0, in=90] (12.6,1.2);

        \draw (8.5,2.2) -- (8.6,3.0);
        \draw (8.6,2.2) -- (8.7,3.0);

        \node (S) at (7.5,3.8) {2};
        \foreach \y in {(S1), (S2), (S3)}
            \draw (S) -- \y;

        \node (K) at (8.3,3.8) {3};
        \foreach \y in {(K1), (K2), (K3)}
            \draw (K) -- \y;

        % Druckausgleich

            \draw (13.,-2) rectangle node {6} (13.75,-1);

            \draw[double] (12,-.2) to [out=180, in=180] (13,-1.3);
            \draw[double] (11.5,.3) to [out=-90, in=180] (13,-1.4); 
            \draw[double](13.3,.9) to [out=-90, in=180] (13,-1.2);

        % Beschriftung
        \node at (0,4) {\parbox{4.5cm}{
                1: Schrittmotor \\
                2: Steckverbindung \\
                3: Kupplung, tropffrei \\
                4: Ultraschallemitter \\
                5: Spiegel \\
                6: Druckausgleichsbehälter
            }
        };
    \end{tikzpicture}
