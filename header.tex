
% Zeilenumbrüche

\parindent 0pt
\parskip 6pt

% Für deutsche Buchstaben und Synthax

\usepackage[ngerman]{babel}

% Für Auflistung mit speziellen Aufzählungszeichen

\usepackage{paralist}

% zB für \del, \dif und andere Mathebefehle

\usepackage{amsmath}
\usepackage{commath}
\usepackage{amssymb}

% Für Literatur/bibliography

\IfFileExists{bibliography.bib}{
    \newcommand{\bibliographyfile}{bibliography.bib}
}{
    \IfFileExists{../../bibliography.bib}{
        \newcommand{\bibliographyfile}{../../bibliography.bib}
    }
}
    

\usepackage[
    backend=biber,
    style=alphabetic,
    hyperref=true
]{biblatex}

\IfFileExists{\bibliographyfile}{
    \bibliography{\bibliographyfile}
    \AtEndDocument{\printbibliography}
}{}

% Für \SIunit[]{} und \num in deutschem Stil

 \usepackage[output-decimal-marker={,}]{siunitx}
 \DeclareSIUnit\clight{\ensuremath{c}}

% Schriftart und encoding

\usepackage[utf8]{inputenc}
% Bitstream charter als default
\usepackage[charter, greekuppercase=italicized]{mathdesign}
% Lato, als sans default
\renewcommand{\sfdefault}{fla}

% Für \sfrac{}{}, also inline-frac

\usepackage{xfrac}

% Für Einbinden von pdf-Grafiken

\usepackage{graphicx}

% Zellen in Tabelle verbinden

\usepackage{multirow}

% einzelne Querformat-Seiten

\usepackage{pdflscape}

% TikZ

\usepackage{tikz}
\usetikzlibrary{arrows}
\usetikzlibrary{calc}

% Umfließen von Bildern

% \usepackage{floatflt}

% Für weitere Farben

\usepackage{color}

% Für Streichen von z.B. $\rightarrow$

\usepackage{centernot}

% Für Befehl \cancel{}

\usepackage{cancel}
\newcommand\ccancel[2][black]{\renewcommand\CancelColor{\color{#1}}\cancel{#2}}

% Für Links nach außen und innerhalb des Dokumentes

\usepackage{hyperref}

% Für Layout von Links

\hypersetup{
	citecolor=black,
	colorlinks=true,
	linkcolor=black,
	urlcolor=blue,
}

% Verschiedene Mathematik-Hilfen

% Manuelles taggen z.B. in align*-Umgebung

\newcommand\mantag{\stepcounter{equation}\tag{\theequation}}

\newcommand \e[1]{\cdot10^{#1}}
\newcommand\p{\partial}

\newcommand\half{\frac 12}
\newcommand\shalf{\sfrac12}

\newcommand\skp[2]{\left\langle#1,#2\right\rangle}
\newcommand\mw[1]{\left\langle#1\right\rangle}

\newcommand \ee{\mathrm e}
\newcommand \eexp[1]{\mathrm{e}^{#1}}
\newcommand \dexp[1]{\exp\left({#1}\right)}

% Trigonometrische Funktionen mit Argument in Klammern

\newcommand \dsin[1]{\sin\left({#1}\right)}
\newcommand \dcos[1]{\cos\left({#1}\right)}
\newcommand \dtan[1]{\tan\left({#1}\right)}
\newcommand \darccos[1]{\arccos\left({#1}\right)}
\newcommand \darcsin[1]{\arcsin\left({#1}\right)}
\newcommand \darctan[1]{\arctan\left({#1}\right)}

\newcommand{\ui}[1]{\int_{-\infty}^{\infty}\dif {#1}\;}

% Für fette, serifenlose Matrix

\newcommand \mat[1]{\mathbf{#1}}

% Nabla und Kombinationen von Nabla

\renewcommand\div[1]{\skp{\nabla}{#1}}
\newcommand\rot{\nabla\times}
\newcommand\grad[1]{\nabla#1}
\newcommand\laplace{\triangle}
\newcommand\dalambert{\mathop{{}\Box}\nolimits}

%Für komplexe Zahlen

\newcommand \ii{\mathrm i}
\renewcommand{\Im}{\mathop{{}\mathrm{Im}}\nolimits}
\renewcommand{\Re}{\mathop{{}\mathrm{Re}}\nolimits}

%Für Bra-Ket-Notation

\newcommand\bra[1]{\left\langle#1\right|}
\newcommand\ket[1]{\left|#1\right\rangle}
\newcommand\braket[2]{\left\langle#1\left.\vphantom{#1 #2}\right|#2\right\rangle}
\newcommand\braopket[3]{\left\langle#1\left.\vphantom{#1 #2 #3}\right|#2\left.\vphantom{#1 #2 #3}\right|#3\right\rangle}
